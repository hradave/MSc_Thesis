%%% Intro.tex --- 
%% 
%% Filename: Intro.tex
%% Description: 
%% Author: Ola Leifler
%% Maintainer: 
%% Created: Thu Oct 14 12:54:47 2010 (CEST)
%% Version: $Id$
%% Version: 
%% Last-Updated: Thu May 19 14:12:31 2016 (+0200)
%%           By: Ola Leifler
%%     Update #: 5
%% URL: 
%% Keywords: 
%% Compatibility: 
%% 
%%%%%%%%%%%%%%%%%%%%%%%%%%%%%%%%%%%%%%%%%%%%%%%%%%%%%%%%%%%%%%%%%%%%%%
%% 
%%% Commentary: 
%% 
%% 
%% 
%%%%%%%%%%%%%%%%%%%%%%%%%%%%%%%%%%%%%%%%%%%%%%%%%%%%%%%%%%%%%%%%%%%%%%
%% 
%%% Change log:
%% 
%% 
%% RCS $Log$
%%%%%%%%%%%%%%%%%%%%%%%%%%%%%%%%%%%%%%%%%%%%%%%%%%%%%%%%%%%%%%%%%%%%%%
%% 
%%% Code:


\chapter{Introduction}
\label{cha:introduction}

\section{Motivation}
\label{sec:motivation}


It is estimated that around 300,000 new brain and nervous system cancer cases occurred in 2020 worldwide. Around 250,000 deaths occurred from this type of cancer in the same year (\cite{cancer_stats2020}). The World Health Organization classifies tumors into grades based on their malignancy, where grade I is the least malignant and grade IV is the most malignant (\cite{louis}). Grade II and III cancers are called Lower Grade Gliomas (LGG), and grade IV cancers are called Glioblastoma or Glioblastome Multiforme (GBM) (\cite{ostrom}).

It is important to diagnose cancer types correctly, because treatment options and survival expectancy depend largely on how malignant a tumor is and what characteristics it has. There are histological differences between different types, which helps the expert pathologist in the decision making. Grade I lesions have the possibility of cure after surgery alone, grade II tumors are more infiltrative, can progress to higher grades, and often recur, and grade III is reserved for cancer that has some evidence of malignancy. The treatment of grade III lesions usually include radiation and chemotherapy. Grade IV tumors are malignant, active, necrosis-prone (death of the tissue), progress quickly and often cause fatality (\cite{cancer_grades2007}).

Histology is a branch of biology concerned with biological tissues, and its aim is to discover structures and patterns of cells and intercellular substances. Histologists examine tissue samples that have been removed from the living body through surgery or biopsy. These samples are processed and stained with chemical dyes to make the structures more visible. They are then cut into very thin slices that can be placed under an optical microscope and examined further.  \cite{histology_def}

Pathology is a medical branch, where experts aim to determine the causes of diseases \cite{pathology_def}, and histopathology connects the two fields, by studying the diseases of tissues under a microscope. Histopathologists make diagnoses based on tissues and help clinicians in the decision making process. Specifically, they often provide diagnostic services for cancer, by reporting its malignancy, grade and possible treatment options \cite{histopathology_def}.

With the advance of technology, it is now possible to scan, save, analyze and share tissue images using virtual microscopy. This technology scans a complete microscope slide and creates a single high resolution file called Whole Slide Image (WSI). These files take up significant storage and require specific software to view and manipulate them, because they are stored in special file formats \cite{wsi_def}.

In this paper, Whole Slide Images from The Cancer Genome Atlas (TCGA) are used. The dataset is publicly available, and contains tissues from GBM and LGG brain cancer types from many different clinics. There are 860 examples of GBM and 844 examples of LGG available as Diagnostic Slides. The images are labeled as a whole, therefore no pixel-wise annotation is available. The files can be more than 3 GB in size, and their resolution is ofter higher, than 100,000 x 100,000. This is why they are saved in a special format (svs), that allows for storage of such large files.

The images were scanned at multiple different resulutions, which can be separately obtained thanks to the special file format. Not all images have the same highest magnification level, however. All LGG classified scans were recorded at 40x magnification (0.25 $\mu$m/pixel), while 77\% of GBM scans have only 20x magnification (0.5 $\mu$m/pixel) available. In order to analyze them together, all images need to be obtained at the same level.

Since the images are so large, it is impossible to process them as a whole, therefore patches or tiles are extracted from them, that are easier to handle for a neural network.

\section{Aim}
\label{sec:aim}


The aim of this thesis is to classify two different types of brain tumor (GBM and LGG) from Whole Slide histology images using Deep Learning. 

\section{Research questions}
\label{sec:research-questions}

This paper intends to find the answer to following complex research question:

What is the best approach with Deep Learning for GBM vs LGG classification using histology images without pixel level annotation?

The competing approaches are compared using statistical methods. There are several challenges regarding the research question, one of which is that the slide images are large in size, therefore they need to be divided into patches. They also come from different sources, so they must be normalized. There is no annotation available on a pixel level, so it is possible that patches in a slide are cancer-free. The patches need to be combined to slide level in the prediction phase, which is not a straight-forward task.


\section{Delimitations}
\label{sec:delimitations}

This is where the main delimitations are described. For
example, this could be that one has focused the study on a
specific application domain or target user group. In the
normal case, the delimitations need not be justified.

%\nocite{scigen}
%We have included Paper \ref{art:scigen}

%%%%%%%%%%%%%%%%%%%%%%%%%%%%%%%%%%%%%%%%%%%%%%%%%%%%%%%%%%%%%%%%%%%%%%
%%% Intro.tex ends here


%%% Local Variables: 
%%% mode: latex
%%% TeX-master: "demothesis"
%%% End: 
